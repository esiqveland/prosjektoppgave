\clearpage
\section{Results}
\label{sec:experiments}

Here follows our results from the testing.

\subsection{Maximum throughput}
We started out by looking at how many queries we could answer with the cluster, i.e. maximum throughput.

In order to find this we use a load generator which creates random queries and send these at various intervals to the cluster. As the traffic is UDP we expect to see a drop in answers from once the system is over saturated. The load generator is run on a more powerful computer outside the cluster.

We do this first by sending queries through a proxy of sorts, the load distributor, which chooses a random worker node to send the query to, then the worker sends the result back to the original client.
Later, we look at exposing direct worker access at the client.

\subsubsection{1 load distributor and 7 workers}
From the numbers in Table \ref{tab:cluster_load_dir} we see that there is a clear drop in performance at around 4300 requests per second. If we increase the load from this point we see a drastic loss in received responses. At this rate the load distributor is running on close to 100\% CPU while the workers are running at around 80\%. So we have reason to believe that the load distributor is holding back the system, effectively being the bottleneck.

\pgfplotstableread{../datasets/cluster_load_dir_requests.txt}\clusterloaddir
\begin{table}[h]
	\centering
	\pgfplotstabletypeset[
     	columns={requests, received},
     	every head row/.style={before row=\hline,
     	after row=\hline},
		every last row/.style={after row=\hline},
		columns/requests/.style={column name=Queries per second},
		columns/received/.style={column name=\% queries served},
     	]
    {\clusterloaddir}
	\caption{Maximum throughput with load distributor}
	\label{tab:cluster_load_dir}
\end{table}

\subsubsection{Required amount of nodes to deliver maximum throughput}
Since the load distributor appears to be a bottleneck in our system, it could be interesting to see how many workers we need to still be able to perform at maximum throughput, i.e. to see how many workers we can drop and still deliver the same amount of work.
In this experiment we have a look at the CPU utilization and query answer rate of the workers while gradually reducing the amount of workers in the cluster.

As suspected, when shutting down nodes one at a time and plotting performance at peak load, we first see an increase in dropped packets after 3 workers have been removed.

The full results are given in Table \ref{tab:clusterreduced}. We see that the cluster could run with 99.99\% of queries answered with 5 instead of 7 workers, and at 84.78\% answered queries with 4. With less than 3 workers, system performance degrades quickly with 58.91\% of queries answered at 3 workers.

We are quite happy with these results as they show that the work distribution looks fairly even. If we only needed a few nodes to handle the maximum throughput from the load distributor, there would be no use for 8 nodes in the cluster.

\pgfplotstableread{../datasets/cluster_reduced_workers_at_full_load.txt}\clusterreduced
\begin{table}
	\centering
	\pgfplotstabletypeset[
     	columns={workers, received},
     	every head row/.style={before row=\hline,
     	after row=\hline},
		every last row/.style={after row=\hline},
		columns/workers/.style={column name=Active working nodes},
		columns/received/.style={column name=\% queries served},
     	]
    {\clusterreduced}
	\caption{Performance when reducing working nodes}
\label{tab:clusterreduced}
\end{table}

\subsubsection{Power usage}
From similar work\cite{RPI_BEOWULF} we have seen that the Raspberry PI will drain up to 15\% more power under heavy load.

\pgfplotstableread{../datasets/watt_per_node.txt}\wattpernode
\begin{table}
	\centering
	\pgfplotstabletypeset[
     	columns={nodes, watt},
     	every head row/.style={before row=\hline,
     	after row=\hline},
		every last row/.style={after row=\hline},
		columns/nodes/.style={column name=Active Nodes},
		columns/watt/.style={column name=Watt},
     	]
    {\wattpernode}
	\caption{Watts consumed under load per node}
    \label{tab:wattpernode}
\end{table}

Despite having read that the Raspberry PIs would drain up to 15\% more energy under load, we were unable to have them drain more than about 9\% more. The cluster runs idle at 22-23W and 23-24W under load. These measurements include the switch which consumes a constant 4W. The power adapter that converts the power for the cluster also has a constant drain of 4W. This leads to the high initial drain we see for the first node.

One source of error for us is that the resolution on the measuring tool is not very high, it does not show decimals, and we do not know how rounding is handled.

\begin{figure}[!h]
\centering
	\begin{tikzpicture}
	\pgfplotstableread{../datasets/watt_per_node.txt}\wattpernode

	\begin{axis}
	[
	xlabel = active nodes,
	xmax = 9,
	xmin = 0,
	ylabel = consumption (W),
	ymax = 30,
	ymin = 0
	]
	\addplot table[y = watt] from \wattpernode ;
	\end{axis}
	\end{tikzpicture}
	\caption{Plot of power drain in the cluster. Shows linear scaling and the fairly high startup cost.}
\end{figure}

\subsubsection{Varying amount of nodes in the cluster}
We also want to vary the amount of nodes that are active in the cluster and plot performance and power consumption to see how it scales.
This is one of the more interesting experiments as one of the strengths of this cluster is that it is easy to power down nodes during hours of low load. We did not get to implement this feature, but it can be done. We will also plot the amount of queries we can deliver per watt used with differing numbers of nodes running.

As we can see from Table \ref{tab:cluster_reqwattnode} we have a steady increase in requests we can service per watt every time we add a node to the cluster until we reach five workers. At this point the load distributor is at max capacity and we are unable to utilize the works fully.

\begin{figure}[!h]
\centering
	\begin{tikzpicture}
	\pgfplotstableread{../datasets/cluster_load_dist_request_watt_per_node.txt}\clusterreqwattnode

	\begin{axis}
	[
	xlabel = active nodes,
	xmax = 9,
	xmin = 0,
	ylabel = Queries per Watt,
	ymax = 200,
	ymin = 0
	]
	\addplot table[y = reqwatt] from \clusterreqwattnode ;
	\end{axis}
	\end{tikzpicture}
	\caption{Plot showing the }
\end{figure}

\begin{table}
	\pgfplotstableread{../datasets/cluster_load_dist_request_watt_per_node.txt}
	\clusterreqwattnode
	\centering
	\pgfplotstabletypeset[
     	columns={nodes,request,	watt, reqwatt},
     	every head row/.style={before row=\hline,
     	after row=\hline},
		every last row/.style={after row=\hline},
		columns/requests/.style={column name=Queries per second},
		columns/watt/.style={column name=Watt},
		columns/reqwatt/.style={column name=Queries per watt},
     	]
    {\clusterreqwattnode}
	\caption{Efficiency with various nodes}
\label{tab:cluster_reqwattnode}
\end{table}

\subsubsection{Discussion}
After the first phase of testing we can draw some interesting conclusions. With regards to throughput there is a clear point where the system can't keep up with the traffic. At this point the load balancers UDP buffers are filling up and we lose queries.

We verified this by looking at the system network buffers while under load. In Linux, one can look at the buffer status with the command $$netstat -aupn$$, giving something like this:
\begin{lstlisting}[caption={Output of {\tt netstat -aupn | grep 32002}},captionpos=b,label={lst:netstat}]
Proto Recv   Send   Local Addres         Address     PID/name
udp   163840 163840 192.168.0.200:32002  0.0.0.0:*   296/bin/load_distri
\end{lstlisting}
In Listing \ref{lst:netstat} we see that the {\tt Recv} buffer and {\tt Send} buffer are currently containing 163840 bytes.
The maximum size of the buffer is given by {\tt /proc/sys/net/core/rmem\_max}. It is listed in bytes and can be modified with a {\tt sysctl} call: {\tt sysctl -w net.core.rmem\_max=value}.
As we can see in this example, the buffer is currently maxed out, and all additional packages will be lost. The load distributor can not handle any more load. Increasing the buffer size only lead to negligible improvements, as the buffer fills up very quickly.

We also see that the performance quickly diminishes as we increase the load. After the breakpoint, practically no additional queries get answered.
This is to be expected.

Our tests regarding energy consumption and scaling show that there is little difference in a Raspberry PI running idle and full load.
It's also of note that the network switch is at 4 watts responsible for 25\% of the consumed power in the cluster.
This would impact further scaling as it would remove the advantage of each PI improving the efficiency of the cluster.

\subsection{8 workers}
As we discovered in phase 1, the load balancer seems to be limiting the throughput of our cluster.
We therefore try to mitigate this by removing the load distributor entirely and rather have the load generating application (client) deal with distributing work.
This would also make it easier to compare with results of the same system running on the MacBook.

By moving the task of distributing load from the cluster and into the load generator (client library) we of course free up one node in the system, so we now have 8 working nodes for these tests. We expect this to improve our throughput and efficiency by a still linear amount, which we later confirm.

\subsubsection{Maximum throughput phase 2}
As we identified the load distributor as the bottleneck in our system we decided to move the load distribution out into the load generator.
This was implemented by having one thread in the client for each node in the cluster, sending queries at various intervals. The results from the load testing are given in Table \ref{tab:cluster_only_workers}.

\pgfplotstableread{../datasets/cluster_only_workers_requests.txt}\clusteronlyworkers
\begin{table}
	\centering
	
	\pgfplotstabletypeset[
     	columns={requests, received},
     	every head row/.style={before row=\hline,
     	after row=\hline},
		every last row/.style={after row=\hline},
		columns/requests/.style={column name=Queries per second},
		columns/received/.style={column name=\% queries served},
     	]
    {\clusteronlyworkers}
    \caption{Maximum throughput with 8 workers}
\label{tab:cluster_only_workers}
\end{table}

\subsubsection{Scalability and energy efficiency}
As the system still consists of the same nodes doing work we don't expect there to be a change in power consumption, however since there now is 8 workers we should see an improvement in how efficient we can serve queries of various loads.
We will run the same test as in phase 1 where we try to find the breakpoint for the system at an increasing amount of nodes working.

\begin{table}
	\pgfplotstableread{../datasets/cluster_only_workers_request_watt_per_node.txt}
	\clusterworkerreqwatt
	\centering
	\pgfplotstabletypeset[
     	columns={nodes,requests, watt, reqwatt},
     	every head row/.style={after row=\hline},
		every last row/.style={after row=\hline},
		columns/nodes/.style={column name=Active nodes},
		columns/requests/.style={column name=Queries per second},
		columns/watt/.style={column name=Watt},
		columns/reqwatt/.style={column name=Queries per watt},
     	]
    {\clusterworkerreqwatt}
    \caption{Efficiency with various nodes without load balancer}
\label{tab:cluster_worker_req_watt}
\end{table}

As we can see in Table \ref{tab:cluster_worker_req_watt} we are able to achieve 242 queries per second per watt.
This is an increase of 54\%.
This follows from now utilizing all the workers fully as well as having one extra node doing work instead of doing load distribution.

\subsubsection{Discussion}
When skipping the load distributor and using only worker nodes we see that it performs better, and a linear increase in performance is indeed observed.
We are now able to service up to around 6500 queries per second. This is an 52.6\% increase from the 7 workers plus 1 load distributer.
Along with the increase in maximum performance we also see that it behaves a lot better under higher loads.
If we compare the two methods we see that at around 5600 requests per second the load balancer is only able to serve around half its requests, while the cluster is answering 100\% of the requests without the load distributor.

\subsection{Mac VS Cluster of PIs}
In this section we will explore how our system performs compared to a MacBook Pro running the same software. We will again try to find the breakpoints of how much traffic the MacBook can handle and how efficient it can serve the requests. We test in two rounds. First only using 1 search process on the MacBook. This would then not utilize both cores on the CPU. Then we run two processes simultaneously and see how well that performs. We will also measure the power consumption on the mac under both idle and heavy load.

\subsubsection{Maximum throughput}
Running with one core we found the maximum throughput to at around 16000 requests per second. We were unable to pinpoint the exact breakpoint because of limits with small amount of time we can sleep a thread. This is almost three times the performance of the cluster.

However this was to be expected with so much more powerful hardware. With two processes running we were able to push it up to 23577 requests per second. At this point we are pushing the limits of what both the network and the MacBook can handle.

\subsubsection{Energy}
Unlike the Raspberry PIs the MacBook really has a lot of power management built into it. We have measured it running idle at around 14 Watts and managed to push it up to 29 Watts while serving requests.
We also see that how many requests we send to it reflect in the amount of power it consumes. The results of this test can be found in Table \ref{tab:mac_energy_1core} and Table \ref{tab:mac_energy_2core}.

\begin{table}
	\pgfplotstableread{../datasets/mac_energy_1core.txt}
	\macenenrgyonecore
	\centering
	\pgfplotstabletypeset[
     	columns={requests, received, watt, reqwatt},
     	every head row/.style={after row=\hline},
		every last row/.style={after row=\hline},
		columns/requests/.style={column name=Queries per second},
		columns/received/.style={column name=\% queries served},
		columns/watt/.style={column name=Watt},
		columns/reqwatt/.style={column name=Queries per watt},
     	]
    {\macenenrgyonecore}
    \caption{MacBook Pro performance and efficiency running on 1 core}
\label{tab:mac_energy_1core}
\end{table}

\begin{table}
	\pgfplotstableread{../datasets/mac_energy_2core.txt}
	\macenergytwocore
	\centering
	\pgfplotstabletypeset[
     	columns={requests, received, watt, reqwatt},
     	every head row/.style={after row=\hline},
		every last row/.style={after row=\hline},
		columns/requests/.style={column name=Queries per second},
		columns/received/.style={column name=\% queries served},
		columns/watt/.style={column name=Watt},
		columns/reqwatt/.style={column name=Queries per watt},
     	]
    {\macenergytwocore}
    \caption{MacBook Pro performance and efficiency running on 2 cores}
\label{tab:mac_energy_2core}
\end{table}

It is apparent that under maximum load not surprisingly there is no way the Pis can contest the MacBook Pro on anything other that startup cost price.
However it is interesting to see how our system compares on lower request rates compared to the MacBook. 
At request rates that are within the limits of our system, the system performs comparable to the MacBook. One could argue that the cost of a MacBook Pro system is comparable when comparing performance. A MacBook Pro runs at about 4 times what it cost to build the cluster. It is apparent however that we could never match the MacBook on power efficiency.

\subsubsection{Cold reads}
All the other experiments have been performed on the cluster or MacBook Pro after a significant number of queries already has been answered. This ensured replicate-able results as we assume most the query words are already in memory and cache. However it would be interesting to see how the two compares when answering queries right after a reboot. In this experiment we send 5000 queries to each core operating, so in total 10000 for the MacBook and 40000 for the cluster. We then try pinpoint the breakpoint where the machines are unable to keep up with the load. The Pi cluster is run in the mode where all nodes are workers.

When measuring these numbers we ran into some interesting problems. For one, the sleep system call we use to limit the interval between queries is limited at milliseconds. This turned out to be a problem for measuring the performance of the MacBook Pro as its breakpoint lies somewhere between the minimum sleep duration and no sleep duration. This is reflected in table\ref{tab:coldread_mac} where we have a gap between a load we clearly can support and a load we clearly cannot support.

\begin{table}[h]
	\pgfplotstableread{../datasets/coldread_pi.txt}
	\coldreadpi
	\centering
	\pgfplotstabletypeset[
     	columns={orps, oanswers},
     	every head row/.style={after row=\hline},
		every last row/.style={after row=\hline},
		columns/orps/.style={column name=Queries per second},
		columns/oanswers/.style={column name=received(\%)},
     	]
    {\coldreadpi}
    \caption{Cold reads Pi cluster. Queries per second, and results received in \%.}
\label{tab:coldread_pi}
\end{table}

\begin{table}[h]
	\pgfplotstableread{../datasets/coldread_mac.txt}
	\coldreadmac
	\centering

	\pgfplotstabletypeset[
     	columns={orps, oanswers},
     	every head row/.style={after row=\hline},
		every last row/.style={after row=\hline},
		columns/orps/.style={column name=Queries per second},
		columns/oanswers/.style={column name=received(\%)},
     	]
    {\coldreadmac}
    \caption{Cold reads MacBook. Queries per second, and results received in \%.}
\label{tab:coldread_mac}
\end{table}

Without optimization the Pi cluster is able to support around 4425 queries per second. This is 67\% of the load it can handle after it has been running for a while.

The MacBook Pro performed worse. At only around 8000 queries per second the MacBook is severely crippled by all the disk access. 

We did expect the cluster to gain some ground when we forced this much disk access. We believe this is mostly due to the Pis parallel disk access. It is also worth noting that the MacBook Pro is benchmarked with random reads at around 45.7MB/s and sequential reads at 450MB/s. The Raspberry Pis can perform random reads at approximately 2.9-7MB/s and sequential reads at approximately 17MB/s


\subsection{Overclocked Pis}
As it turns out every operation in our system is bounded by CPU speed, it would be reasonable to expect enhanced performance by overclocking the CPU.
By changing from no over clocking to turbo mode we increase the clock frequency from 700MHz to 1000MHz. We also increase the GPU core frequency which drives the L2 cache.

As mentioned in the hardware section, overclocking the Raspberry PI might bring certain downsides. Users have experienced data corruption on the SD card as well as general instability. There are also reports of different degrees of tolerance among the Raspberry PIs. Some works perfectly under any degree of overclocking while others quickly start showing symptoms of instability.

We started out by overclocking one node and then run some tests. Everything seemed to work, so we added the overclocking parameters to the entire cluster. After then rebooting the cluster we experienced all the problems mentioned above. One node had problems with its SD-card and two others were struggling with booting or network access. We did however manage to find a stable set of five nodes to compare with our existing results. In this experiment we ran five individual workers similarly to when we tried to pin point the maximum performance of the cluster with various amounts of nodes.

We were able to see an increase in throughput of around 10\% with the cores overclocked. This number was a lot lower than expected taking into account that the cluster was using 21 watts compared to 16 without the overclock. At this rate the cluster can deliver 133 requests per watt, compared to 152 without the overclock. So this was generally bad value.


\subsection{Experiences with {C} and Python}
The first testings of the C-port showed significant improvements in delivery speed.
It further confirmed that managing memory yourself is desirable for performance, and is especially important in an environment where resources are scarce.
When blasting the searching code with queries, we quickly maxed out the available memory.
These memory leaks were later fixed, and the program now experiences constant memory usage.

The difference in developing platform, with x86\_64 Darwin compared to the ARMv6 runtime environment did not impose any extra difficulties, except for a single case of some odd pointer behavior.
The rest of the time, the code behaved as expected from one platform to the other, for both Python and C.

For comparison between the two languages, we created a test set of 100,000 queries. We then ran this test set against the reference MacBook computer, and on the Pi, both for Python and {C}.
The results are given in Table \ref{tbl:runtimes_ports}.

For the relatively quick {Core i5}, the python code need 10 times longer to run the queries than the C code. This difference in performance seem reasonable compared to what is known about the differences between interpreted dynamic languages and C.
The slower Python code is however detrimental to the performance for the weaker Pi. The same 100,000 queries takes 84 seconds to run in C, but takes close to 2047 seconds in Python, 24 times longer.

\begin{table}[h]
    \begin{center}
    \begin{tabular}{|r|r|r|r|}
    \hline
       & \multicolumn{1}{|c|}{MacBook} & \multicolumn{1}{|c|}{PI (OC)}  & \multicolumn{1}{|c|}{PI} \\
    \hline
    C      & 4.78 & 55.93 & 84.75     \\
    \hline
    Python & 49.41 & 1268.89 & 2046.77   \\

    \hline
    \end{tabular}
    \caption{Runtime (in seconds) for C and Python code on 100,000 queries. OC is the overclocked Pi.}
    \label{tbl:runtimes_ports}
    \end{center}
\end{table}

Further we found a problem with our message generating code. Originally we always sent queries as 1024 byte messages. This was done for simplicity but of course at a cost of performance. By instead sending only as many bytes as needed we saw a dramatic increase in performance.











