\clearpage

\section{Abstract}
Energy consumption and the corresponding heat dissipation in becoming a considerable challenge with regards to computer centers and super computers. Today most large scale computers use powerful multi core CPUs and have tens if not hundreds of gigabytes of RAM. In addition to being very expensive up front in the acquisition there is also a lot of cost associated with operating these computers. We don't have to look any further than the IBM produced Roadrunner which while still being the 22nd fastest computer in the world, was shut down in March 2013 because it was no longer with the power bill related to operating it.\cite{roadrunner}

Much research today is related to making computers more power efficient. Intels Haswell micro architecture brings significant enhancements in laptop efficiency. However these improvements are related to saving power by slowing down the cores whenever theres lack of work. Another interesting area is the mobile processors. ARM processors are very light weight and typically run very efficiently. These cores are simpler and a lot slower than traditional CPUs, however they are very cheap. There has been some research in the use of clusters containing multiple of these ARM computers and performing normal cluster tasks. In this paper we make a cluster of 8 Raspberry PI computers and use them to process queries over a set of documents.


\section{Introduction}
\label{sec:introduction}
This section discusses the background for our project and the goals and means we chose for executing this.
Section \ref{sec:related} briefly explores related work.
In Section \ref{sec:hardware}, we introduce the properties of the hardware we will be using, and look at alternatives.
Section \ref{sec:software} explains the software we have built for this project.
Section \ref{sec:build} shows how we built the hardware of the system we will be using.
Section \ref{sec:experiments} contains the results and discussion of our work.

\subsection{Background}
Small single board system on chip computers are getting increasingly popular. They are tiny, low powered and cheap. But do they pack a punch?
How would a set of these low energy devices compare up to an ordinary desktop computer if we take energy consumption and performance into account, and can they be used for cost efficiently scaling a system to different workloads?

In this report we will try to answer this. Our focus will be on delivering query results from a search system, stressing the two likely slowest parts of the PI:
It's disk access and network.

\subsection{Goals}
To explore the possibilities of more power efficient clusters.

\subsection{Method}
We will be using a small home made search system, and use a load generator to test the system, measuring throughput and power usage against a known desktop system.

